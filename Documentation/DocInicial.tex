\documentclass[a4paper]{article}

%% Language and font encodings
\usepackage[english]{babel}
\usepackage[utf8x]{inputenc}
\usepackage[T1]{fontenc}

%% Sets page size and margins
\usepackage[a4paper,top=3cm,bottom=2cm,left=3cm,right=3cm,marginparwidth=1.75cm]{geometry}

%% Useful packages
\usepackage{amsmath}
\usepackage{graphicx}
\usepackage[colorinlistoftodos]{todonotes}
\usepackage[colorlinks=true, allcolors=blue]{hyperref}

\title{Music recomendation}
\author{}

\begin{document}
\maketitle
\tableofcontents

\section{Descripción del trabajo}


\subsection{Notas sobre el lenguaje de programación escogido}


\section{Trabajo Relacionado}


%\section{Posibles Métodos}
%\begin{itemize}
%\item logistic regression, multinomial regression
%(single-layer MLP), LDA, QDA, RDA, \textbf{Naive Bayes}, \textbf{nearest-neighbours},\textbf{linear SVM}, quadratic SVM
%\item one-hidden-layer MLP, the RBFNN, the SVM with RBF kernel, a
%Random Forest
%\end{itemize}


\section{Data exploration and Preprocessing}

Los datos iniciales son: 
\begin{itemize}
\item train.csv
\item test.csv
\item songs.csv
\item members.csv
\item song\_extra\_info.csv
\end{itemize}

La idea inicial es mejorar train utilizando songs, members y song\_extra\_info y convertirlo en un único train y test. 

Después pasar todas las variables categóricasa numéricas y aplicarle un MCA.

Debería quedar: 
\begin{itemize}
\item clean\_train.csv
\item clean\_test.csv
\end{itemize}


\subsection{Tratamiento de valores perdidos}

\subsection{Tratamiento de outliers}

\subsection{Tratamiento de valores incorrectos}

\subsection{Codificación de variables categóricas}

\subsection{Selección de features}

\subsection{Creación de nuevas variables}

\subsection{Estandarización}

\subsection{Transformación de variables}

\section{Resampling Protocol}

\section{Resultados de los métodos lineales}
logistic regression, multinomial regression
(single-layer MLP), LDA, QDA, RDA, \textbf{Naive Bayes}, \textbf{nearest-neighbours},\textbf{linear SVM}, quadratic SVM
\section{Resultados de los métodos no lineales}
one-hidden-layer MLP, the RBFNN, the SVM with RBF kernel, a
Random Forest
\section{Descripción y justificación del modelo escogido}

\section{Conclusiones}

\end{document}